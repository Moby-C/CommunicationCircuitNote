\chapter{高频小信号放大器}

\section{概述}

高频小信号放大器的特点:放大高频小信号(中心频率在几百kHz到几百 MHz,频谱宽度在几 kHz 到几十 MHz 的范围内)的放大器。

\subsection{高频小信号放大器的分类}

\begin{equation}
    \notag
    \text{高频小信号放大器} \left\{
        \begin{array}{l}
            \text{谐振放大器(窄带)} \left\{
                \begin{array}{l}
                    \text{单振荡回路} \\
                    \text{耦合振荡回路}
                \end{array}
            \right. \\
            \text{非谐振放大器(宽带)} \left\{
                \begin{array}{l}
                    \text{LC集中滤波器} \\
                    \text{石英晶体滤波器} \\
                    \text{陶瓷滤波器} \\
                    \text{声表面波滤波器}
                \end{array}
            \right.
        \end{array}
    \right.
\end{equation}

\subsection{高频小信号放大器的主要质量指标}

\subsubsection{增益(放大系数)}

高频小信号放大器的电压增益及分贝表示:

\begin{equation}
    A_v = \frac{V_o}{V_i}, \quad A_v = 20 \log{\frac{V_o}{V_i}}
\end{equation}

高频小信号放大器的功率增益及分贝表示:

\begin{equation}
    A_p = \frac{P_o}{P_i}, \quad A_p = 10 \log{\frac{P_o}{P_i}}
\end{equation}

\subsubsection{通频带}

高频小信号放大器的通频带为:

\begin{equation}
    2\Delta \omega_{0.7} = \frac{\omega_0}{Q_L}, \quad 2\Delta f_{0.7} = \frac{f_0}{Q_L}
\end{equation}

\subsubsection{选择性}

从各种不同频率信号的总和(有用的和有害的)中选出有用信号,抑制干扰信号的能力称为放大器的选择性。选择性常采用矩形系数和抑制比来表示。

\subsubsection{矩形系数}

矩形系数表示与理想滤波特性的接近程度,定义为:

\begin{equation}
    K_{r0.1} = \frac{B_{0.1}}{B_{0.7}} = \frac{2 \Delta f_{0.1}}{2 \Delta f_{0.7}} = \sqrt{10^2 - 1}
\end{equation}

\subsubsection{抑制比}

表示对某个干扰信号 $f_n$ 的抑制能力,定义为:

\begin{equation}
    d_n = \frac{A_{v_0}}{A_{v_n}}
\end{equation}

其中 $A_{v_0}$ 为 $f_0$ 点输出电压,$A_{v_n}$ 为 $f_n$ 点输出电压。

\subsubsection{工作稳定性}

工作稳定性是指放大器的工作状态(直流偏置)、晶体管参数、电路元件参数等发生可能的变化时,放大器的主要特性的稳定。

\subsection{高频小信号放大器的分析方法}

晶体管工作在线性区,可看成线性元件,可用有源四端网络参数微变等效电路来分析。

\section{晶体管高频小信号等效电路与参数}

\subsection{形式等效电路(网络参数等效电路)}

双口网络的输入导纳、反向传输导纳、正向传输导纳、输出导纳分别为:

\begin{equation}
    y_i = \left. \frac{\dot{I_1}}{\dot{V_1}} \right|_{\dot{V_2} = 0}, \quad 
    y_r = \left. \frac{\dot{I_1}}{\dot{V_2}} \right|_{\dot{V_1} = 0}, \quad 
    y_f = \left. \frac{\dot{I_2}}{\dot{V_1}} \right|_{\dot{V_2} = 0}, \quad 
    y_o = \left. \frac{\dot{I_2}}{\dot{V_2}} \right|_{\dot{V_1} = 0}
\end{equation}

晶体管共发射极放大电路与 $y$ 参数等效电路:

晶体管共发射极放大电路的输入导纳、输出导纳、电压增益分别为:

\begin{equation}
    Y_i = y_{ie} - \frac{y_{re}y_{fe}}{y_{oe} + Y_L}, \quad 
    Y_o = y_{oe} - \frac{y_{re}y_{fe}}{y_{ie} + Y_S}, \quad 
    \dot{A}_v = \frac{\dot{A}_2}{\dot{A}_1} = - \frac{y_{fe}}{y_{oe} + Y_L}
\end{equation}

$y_{fe}$ 越大,表示晶体管的放大能力越强;$y_{re}$ 越大,表示晶体管的内部反馈越强。$y_{re}$ 是谐振放大器自激的根源,应尽可能使其减小。   

$y$ 参数的优点:各个元件在很宽的频率范围内都保持常数。

$y$ 参数的缺点:会随频率变化,物理含义不明显,分析电路不够方便。

\subsection{混合 $\pi$ 等效电路}

混合 $\pi$ 等效电路:



\subsection{混合 $\pi$ 等效电路参数与形式等效电路 $y$ 参数的转换}

对于共发射极放大电路,合 $\pi$ 等效电路参数与形式等效电路 $y$ 参数的转换方式为:

\begin{equation}
\begin{aligned}
    y_i &= \left. \frac{\dot{I_1}}{\dot{V_1}} \right|_{\dot{V_2} = 0} = g_{ie} + \text{j}\omega C_{ie} \approx g_{b'e} + \text{j}\omega (C_{b'e} + C_{b'c}) \\
    y_r &= \left. \frac{\dot{I_1}}{\dot{V_2}} \right|_{\dot{V_1} = 0} = g_{oe} + \text{j}\omega C_{oe} \approx g_{ce} + \text{j}\omega C_{b'c} \\
    y_f &= \left. \frac{\dot{I_2}}{\dot{V_1}} \right|_{\dot{V_2} = 0} \approx \frac{g_m}{1 + \text{j}\omega (C_{b'e} + C_{b'c})r_{bb'}} \approx g_m \\
    y_o &= \left. \frac{\dot{I_2}}{\dot{V_2}} \right|_{\dot{V_1} = 0} \approx - \frac{\text{j}\omega C_{b'c}}{1 + \text{j}\omega C_{b'e} r_{bb'}} \approx - \text{j}\omega C_{b'c}
\end{aligned}
\end{equation}

\subsection{晶体管的高频参数}

\subsubsection{截止频率 $f_{\beta}$}

截止频率是 $\beta$ 下降到低频值 $\beta_0$ 的 $\frac{1}{2}$ 时所对应的频率。

\begin{equation}
    \dot{\beta} = \frac{\beta_0}{ 1 + \text{j} \frac{f}{f_{\beta}} }, \quad \beta = \frac{\beta_0}{ \sqrt{1 + \left(\frac{f}{f_{\beta}}\right)^2 } }
\end{equation}

\subsubsection{特征频率 $f_T$}

$\beta = 1$ 时所对应的频率。

\begin{equation}
    f_T = f_{\beta} \sqrt{\beta_0^2 - 1} \approx \beta_0 f_{\beta} 
\end{equation}

当 $f > f_T$ 时,共发接法的晶体管将不再有电流放大能力,但仍可能有电压增益,而功率增益还可能大于 $1$。

当 $f >> f_T$ 时,可以使用 $\beta f \approx f_T$ 粗略计算电流放大系数。

\subsubsection{最高振荡频率 $f_{\text{max}}$}

晶体管功率增益 $G_p = 1$ 时所对应的频率。

$f>f_{\text{max}}$ 时,晶体管不能得到功率放大。也不能产生振荡。

\section{单调谐回路谐振放大器}

通常需要多级放大器来提供足够高的增益和足够好的选择性,从而为下一级(例如混频和检波)提供性能良好的有用信号。

\subsection{高频小信号放大器分析方法}

高频小信号放大器的电路分析包括:多级分单级、静态分析、动态分析、整合系统四个基本步骤。

\subsubsection{多级分单级}

前级放大器是本级放大器的信号源;后级放大器是本级放大器的负载。

\subsubsection{静态分析}

画出直流等效电路,交流输入信号为零;所有电容开路;所有电感短路。

\subsubsection{动态分析}

画出交流等效电路,有交流输入信号,所有直流量为零;所有大电容短路;所有大电感开路。(谐振回路L、C保留。)

画出交流小信号等效电路。

\subsection{电压增益}

谐振时的电压增益:

\begin{equation}
    A_{v_0} = - \frac{p_1 p_2 y_{\text{fe}}}{G_P'} 
    = - \frac{p_1 p_2 y_{\text{fe}}}{G_P + p_1^2 g_{\text{oe}1} + p_2^2 g_{\text{ie}2}}
\end{equation}

其中接入系数 $p_1 = \dfrac{N_1}{N}$,$p_2 = \dfrac{N_2}{N}$。

阻抗匹配时,可获得最大的电压增益为:

\begin{equation}
    (\dot{A}_{v_0})_{\text{max}} = - \frac{y_{\text{fe}}}{2\sqrt{g_{\text{o}1} g_{\text{i}2}}}
\end{equation}

\subsection{功率增益}

谐振时的功率增益:

\begin{equation}
    A_{p_0} = (A_{p_0})^2 \frac{g_{\text{i}2}}{g_{\text{i}1}}
\end{equation}

阻抗匹配时,可获得最大的功率增益为:

\begin{equation}
    (A_{p_0})_{\text{max}} = \frac{|y_{\text{fe}}|^2}{4 g_{\text{o}1} g_{\text{i}2}}
\end{equation}

若考虑谐振回路损耗,其插入损耗为:

\begin{equation}
    K_1 = \left(\frac{1}{1 - \frac{Q_L}{Q_0}}\right)^2
\end{equation}

匹配最大功率增益:

\begin{equation}
    (A_{p_0})_{\text{max}} = \frac{|y_{\text{fe}}|^2}{4 g_{\text{o}1} g_{\text{i}2}} \left(1 - \frac{Q_L}{Q_0}\right)^2
\end{equation}

电压增益为:

\begin{equation}
    (A_{v_0})_{\text{max}} = - \frac{|y_{\text{fe}}|}{2\sqrt{g_{\text{o}1} g_{\text{i}2}}} \left(1 - \frac{Q_L}{Q_0}\right)
\end{equation}

\subsection{通频带与选择性}

单调谐回路谐振放大器通频带为:

\begin{equation}
    2 \Delta f_{0.7} = \frac{f_0}{Q_L}
\end{equation}

选择性用矩形系数表示,即:

\begin{equation}
    K_{r0.1} = \frac{2 \Delta f_{0.1}}{2 \Delta f_{0.7}}
\end{equation}

\subsection{级间耦合网络}

图:单调谐放大器的级间耦合网络形式

\section{多级单调谐回路谐振放大器}

如果各级放大器是由完全相同,则总增益为:

\begin{equation}
    \dot{A}_{v} = \prod{\dot{A}_{v_n}} = \left(\dot{A}_{v_1}\right)^n
\end{equation}

通频带为:

\begin{equation}
    2 \Delta f_{0.7} = \sqrt{2^{\frac{1}{n}} - 1} \frac{f_0}{Q_0}
\end{equation}

矩形系数为:

\begin{equation}
    K_{r0.1} = \sqrt{\frac{100^{\frac{1}{n}} - 1}{2^{\frac{1}{n}} - 1}}
\end{equation}

\section{双调谐回路谐振放大器}

临界耦合 ($\eta = 1$) 时,有:

\begin{equation}
    \frac{A_v}{A_{v_0}} = \frac{2}{\sqrt{4 + \xi^4}}
\end{equation}

其中 $\xi = Q_L \dfrac{2 \Delta f}{f_0}$。

临界耦合时的通频带:

\begin{equation}
    2 \Delta f_{0.7} = \sqrt{2} \frac{f_0}{Q_L}
\end{equation}

矩形系数为:

\begin{equation}
    K_{r0.1} = \sqrt[4]{\frac{100^{\frac{1}{n}} - 1}{2^{\frac{1}{n}} - 1}}
\end{equation}

\section{谐振放大器的稳定性与稳定措施}

\subsection{谐振放大器的稳定性}

放大器的输入和输出导纳:

\begin{equation}
    Y_i = y_{ie} - \frac{y_{re}y_{fe}}{y_{oe} + Y_L}, \quad 
    Y_o = y_{oe} - \frac{y_{re}y_{fe}}{y_{ie} + Y_S}
\end{equation}

自激振荡:$g_{\Sigma} = g_s+ g_{ie} + g_F = 0$,即整个回路的能量消耗为零,回路中储存的能量恒定,在电感与电容之间相互转换,回路中的等幅振荡得以维持,而不需外加激励。

如果反馈电导为负值,则可能发生自激振荡现象。

自激产生的条件:幅值条件、相位条件。

稳定系数:

\begin{equation}
    S = \frac{(g_s + g_{ie})(g_{oe} + g_L)(1 + \xi^2)}{|y_{fe}||y_{re}|}
\end{equation}

如果稳定系数 $S = 1$,放大器可能产生自激振荡。$S$ 越大,放大器工作就越稳定。

\subsection{单向化}

避免自激的做法有中和法和失配法。

失配法:信号源内阻不与晶体管输入阻抗匹配;晶体管输出端负载阻抗不与本级晶体管的输出阻抗匹配。

失配法以牺牲增益为代价换取稳定性的提高。

